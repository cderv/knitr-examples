\documentclass{article}\usepackage[]{graphicx}\usepackage[]{xcolor}
% maxwidth is the original width if it is less than linewidth
% otherwise use linewidth (to make sure the graphics do not exceed the margin)
\makeatletter
\def\maxwidth{ %
  \ifdim\Gin@nat@width>\linewidth
    \linewidth
  \else
    \Gin@nat@width
  \fi
}
\makeatother

\definecolor{fgcolor}{rgb}{0.345, 0.345, 0.345}
\newcommand{\hlnum}[1]{\textcolor[rgb]{0.686,0.059,0.569}{#1}}%
\newcommand{\hlstr}[1]{\textcolor[rgb]{0.192,0.494,0.8}{#1}}%
\newcommand{\hlcom}[1]{\textcolor[rgb]{0.678,0.584,0.686}{\textit{#1}}}%
\newcommand{\hlopt}[1]{\textcolor[rgb]{0,0,0}{#1}}%
\newcommand{\hlstd}[1]{\textcolor[rgb]{0.345,0.345,0.345}{#1}}%
\newcommand{\hlkwa}[1]{\textcolor[rgb]{0.161,0.373,0.58}{\textbf{#1}}}%
\newcommand{\hlkwb}[1]{\textcolor[rgb]{0.69,0.353,0.396}{#1}}%
\newcommand{\hlkwc}[1]{\textcolor[rgb]{0.333,0.667,0.333}{#1}}%
\newcommand{\hlkwd}[1]{\textcolor[rgb]{0.737,0.353,0.396}{\textbf{#1}}}%
\let\hlipl\hlkwb

\usepackage{framed}
\makeatletter
\newenvironment{kframe}{%
 \def\at@end@of@kframe{}%
 \ifinner\ifhmode%
  \def\at@end@of@kframe{\end{minipage}}%
  \begin{minipage}{\columnwidth}%
 \fi\fi%
 \def\FrameCommand##1{\hskip\@totalleftmargin \hskip-\fboxsep
 \colorbox{shadecolor}{##1}\hskip-\fboxsep
     % There is no \\@totalrightmargin, so:
     \hskip-\linewidth \hskip-\@totalleftmargin \hskip\columnwidth}%
 \MakeFramed {\advance\hsize-\width
   \@totalleftmargin\z@ \linewidth\hsize
   \@setminipage}}%
 {\par\unskip\endMakeFramed%
 \at@end@of@kframe}
\makeatother

\definecolor{shadecolor}{rgb}{.97, .97, .97}
\definecolor{messagecolor}{rgb}{0, 0, 0}
\definecolor{warningcolor}{rgb}{1, 0, 1}
\definecolor{errorcolor}{rgb}{1, 0, 0}
\newenvironment{knitrout}{}{} % an empty environment to be redefined in TeX

\usepackage{alltt}
\usepackage{subfig}

\title{Arrange Plots in Columns}
\author{Yihui Xie}
\IfFileExists{upquote.sty}{\usepackage{upquote}}{}
\begin{document}
\maketitle



Three plots in 2 columns via the chunk option \texttt{fig.ncol = 2}. Note that you also need subcaptions (\texttt{fig.subcap}). See Figure \ref{fig:two-and-one}.

\begin{knitrout}
\definecolor{shadecolor}{rgb}{0.969, 0.969, 0.969}\color{fgcolor}\begin{figure}
\subfloat[One\label{fig:two-and-one-1}]{\includegraphics[width=0.48\textwidth]{figure/119-subfig-columns-two-and-one-1} }
\subfloat[Two\label{fig:two-and-one-2}]{\includegraphics[width=0.48\textwidth]{figure/119-subfig-columns-two-and-one-2} }\newline
\subfloat[Three\label{fig:two-and-one-3}]{\includegraphics[width=0.48\textwidth]{figure/119-subfig-columns-two-and-one-3} }\caption[Three plots in two columns]{Three plots in two columns.}\label{fig:two-and-one}
\end{figure}

\end{knitrout}

Manually specify the separators between two plots, e.g., add a new line after the first plot. See Figure \ref{fig:one-and-two}.

\begin{knitrout}
\definecolor{shadecolor}{rgb}{0.969, 0.969, 0.969}\color{fgcolor}\begin{figure}
\subfloat[One\label{fig:one-and-two-1}]{\includegraphics[width=0.48\textwidth]{figure/119-subfig-columns-one-and-two-1} }1st\newline
\subfloat[Two\label{fig:one-and-two-2}]{\includegraphics[width=0.48\textwidth]{figure/119-subfig-columns-one-and-two-2} }2nd
\subfloat[Three\label{fig:one-and-two-3}]{\includegraphics[width=0.48\textwidth]{figure/119-subfig-columns-one-and-two-3} }3rd\caption[Three plots, with the first plot on its own line]{Three plots, with the first plot on its own line.}\label{fig:one-and-two}
\end{figure}

\end{knitrout}


\end{document}
